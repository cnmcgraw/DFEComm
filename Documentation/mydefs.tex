%
%  mydefs.tex   
%  Command definitions that can be used in all documents that have
%      %
%  mydefs.tex   
%  Command definitions that can be used in all documents that have
%      %
%  mydefs.tex   
%  Command definitions that can be used in all documents that have
%      %
%  mydefs.tex   
%  Command definitions that can be used in all documents that have
%      \input{mydefs}
%

%Load the myriad packages
%\usepackage{amssymb,amsmath}
\usepackage{amsfonts}
\usepackage{amsmath}
\usepackage{textcomp}
%\usepackage{graphicx}
\usepackage{grffile}
\usepackage{tikz}
%\usepackage{subeqn} %for eqn 1a, 1b, etc
\usepackage{subfig}    % for multi-figure figures
%\usepackage[numbers, super]{natbib}
%\usepackage{pdflscape}
%\usepackage{rotating}
%\usepackage{grffile} %spaces in file names
%\usepackage{parskip}
%\usepackage[T1]{fontenc} %for sc and bf
%\usepackage{wasysym}
%\usepackage{bigstrut}
%\usepackage{hyperref}
%\usepackage{Float}
\usepackage{multirow}
\usepackage[numbers,sort&compress]{natbib}

%Tikz libraries
\usetikzlibrary{calc,trees,positioning,arrows,chains,shapes.geometric,%
    decorations.pathreplacing,decorations.pathmorphing,shapes,%
    matrix,shapes.symbols}

% Optional for code samples
\usepackage{listings}
\lstloadlanguages{Matlab}
\lstset{language=Matlab,commentstyle=\color{blue},keywordstyle=\color{red}}

%MCNP syntax highlighting
\definecolor{lightgrey}{rgb}{0.9,0.9,0.9}
\definecolor{darkgreen}{rgb}{0,0.6,0}
 
\lstdefinelanguage{MCNP}
{morekeywords={kcode,ksrc,mode,print,prdmp},
sensitive=false,
morecomment=[f]{c},
morecomment=[l]{$},
morestring=[b]",
numbers=left,
numberstyle =\tiny,
}

\lstset{
language=Matlab,
numbers=left,
numberstyle =\tiny,}

%placeholder cite
\newcommand{\CITE}{({\bf cite})}
%singular
\newcommand{\fref}[1]{Fig.~\ref{fig:#1}}
\newcommand{\Fref}[1]{Figure~\ref{fig:#1}}
\newcommand{\eref}[1]{Eq.~(\ref{eq:#1})}
\newcommand{\Eref}[1]{Equation~(\ref{eq:#1})}
\newcommand{\tref}[1]{Table~\ref{tab:#1}}
%plural
\newcommand{\frefs}[2]{Figs.~\ref{fig:#1} and \ref{fig:#2}}
\newcommand{\Frefs}[2]{Figures~\ref{fig:#1} and \ref{fig:#2}}
\newcommand{\erefs}[2]{Eqs.~(\ref{eq:#1}) and (\ref{eq:#2})}
\newcommand{\Erefs}[2]{Equations~(\ref{eq:#1}) and (\ref{eq:#2})}
\newcommand{\trefs}[2]{Tables~\ref{tab:#1} and \ref{tab:#2}}
%range
\newcommand{\frefss}[2]{Figs.~\ref{fig:#1} - \ref{fig:#2}}
\newcommand{\Frefss}[2]{Figures~\ref{fig:#1} - \ref{fig:#2}}
\newcommand{\erefss}[2]{Eqs.~(\ref{eq:#1}) - (\ref{eq:#2})}
\newcommand{\Erefss}[2]{Equations~(\ref{eq:#1}) - (\ref{eq:#2})}
\newcommand{\trefss}[2]{Tables~\ref{tab:#1} - \ref{tab:#2}}
%misc.
\newcommand{\nn}[1]{\ensuremath{^{#1}}} %[1] is # of commands
\newcommand{\keff}{\ensuremath{{k_\mathrm{eff}}}}
\newcommand{\kinf}{\ensuremath{{k_\infty}}}
\newcommand{\alphaT}{\ensuremath{{\alpha_{_T}}}}
\newcommand{\SN}{\ensuremath{{\text{S}_n}}}
\newcommand{\PN}{\ensuremath{{\text{P}_n}}}
\newcommand{\order}[1]{\ensuremath{\mathcal{O}\left(#1\right)}}
% some simplifying commands
\newcommand{\eg}{e.g.}
%\newcommand{\eg}{{\it e.g.}} 
\newcommand{\ie}{i.e.}
\newcommand{\etal}{et al.}
\newcommand{\viz}{viz.}
\newcommand{\cf}{cf.}
\newcommand{\acite}[1]{{\bf(Add Citation: #1)}}
\newcommand{\E}{\mathcal{E}}
% derivative - d
\newcommand{\ud}[1]{\,\mathrm{d}{#1}\;}
% bold unit vector n-hat
\newcommand{\nhat}{\hat{\bf n}}
\newcommand{\tensor}[1]{\mathcal{#1}}
\renewcommand{\vec}[1]{\mathbf{#1}}
%Use for vectors of symbols (or use \pmb)
\newcommand{\vecsym}[1]{\boldsymbol{#1}}
\newcommand{\om}{\boldsymbol{\Omega}}

\newlength \figwidth
\setlength \figwidth {0.8\textwidth}
\captionsetup[subfigure]{width=0.4\textwidth} %trouble-maker
\captionsetup[subfigure]{font={footnotesize,stretch=1.1},skip=0pt}

 
 %LaTeX hyphenation
 \hyphenation{epi-thermal}

%Tikz flowchart parameters

\tikzstyle{block} = [rectangle, draw, fill=white, 
    text width=6em, text centered, rounded corners, minimum height=3em]
\tikzstyle{line} = [draw, thick, -latex']
 \tikzstyle{data} = [text width = 5em, text centered, node distance=1.6cm, inner sep=5pt] 
%

%Load the myriad packages
%\usepackage{amssymb,amsmath}
\usepackage{amsfonts}
\usepackage{amsmath}
\usepackage{textcomp}
%\usepackage{graphicx}
\usepackage{grffile}
\usepackage{tikz}
%\usepackage{subeqn} %for eqn 1a, 1b, etc
\usepackage{subfig}    % for multi-figure figures
%\usepackage[numbers, super]{natbib}
%\usepackage{pdflscape}
%\usepackage{rotating}
%\usepackage{grffile} %spaces in file names
%\usepackage{parskip}
%\usepackage[T1]{fontenc} %for sc and bf
%\usepackage{wasysym}
%\usepackage{bigstrut}
%\usepackage{hyperref}
%\usepackage{Float}
\usepackage{multirow}
\usepackage[numbers,sort&compress]{natbib}

%Tikz libraries
\usetikzlibrary{calc,trees,positioning,arrows,chains,shapes.geometric,%
    decorations.pathreplacing,decorations.pathmorphing,shapes,%
    matrix,shapes.symbols}

% Optional for code samples
\usepackage{listings}
\lstloadlanguages{Matlab}
\lstset{language=Matlab,commentstyle=\color{blue},keywordstyle=\color{red}}

%MCNP syntax highlighting
\definecolor{lightgrey}{rgb}{0.9,0.9,0.9}
\definecolor{darkgreen}{rgb}{0,0.6,0}
 
\lstdefinelanguage{MCNP}
{morekeywords={kcode,ksrc,mode,print,prdmp},
sensitive=false,
morecomment=[f]{c},
morecomment=[l]{$},
morestring=[b]",
numbers=left,
numberstyle =\tiny,
}

\lstset{
language=Matlab,
numbers=left,
numberstyle =\tiny,}

%placeholder cite
\newcommand{\CITE}{({\bf cite})}
%singular
\newcommand{\fref}[1]{Fig.~\ref{fig:#1}}
\newcommand{\Fref}[1]{Figure~\ref{fig:#1}}
\newcommand{\eref}[1]{Eq.~(\ref{eq:#1})}
\newcommand{\Eref}[1]{Equation~(\ref{eq:#1})}
\newcommand{\tref}[1]{Table~\ref{tab:#1}}
%plural
\newcommand{\frefs}[2]{Figs.~\ref{fig:#1} and \ref{fig:#2}}
\newcommand{\Frefs}[2]{Figures~\ref{fig:#1} and \ref{fig:#2}}
\newcommand{\erefs}[2]{Eqs.~(\ref{eq:#1}) and (\ref{eq:#2})}
\newcommand{\Erefs}[2]{Equations~(\ref{eq:#1}) and (\ref{eq:#2})}
\newcommand{\trefs}[2]{Tables~\ref{tab:#1} and \ref{tab:#2}}
%range
\newcommand{\frefss}[2]{Figs.~\ref{fig:#1} - \ref{fig:#2}}
\newcommand{\Frefss}[2]{Figures~\ref{fig:#1} - \ref{fig:#2}}
\newcommand{\erefss}[2]{Eqs.~(\ref{eq:#1}) - (\ref{eq:#2})}
\newcommand{\Erefss}[2]{Equations~(\ref{eq:#1}) - (\ref{eq:#2})}
\newcommand{\trefss}[2]{Tables~\ref{tab:#1} - \ref{tab:#2}}
%misc.
\newcommand{\nn}[1]{\ensuremath{^{#1}}} %[1] is # of commands
\newcommand{\keff}{\ensuremath{{k_\mathrm{eff}}}}
\newcommand{\kinf}{\ensuremath{{k_\infty}}}
\newcommand{\alphaT}{\ensuremath{{\alpha_{_T}}}}
\newcommand{\SN}{\ensuremath{{\text{S}_n}}}
\newcommand{\PN}{\ensuremath{{\text{P}_n}}}
\newcommand{\order}[1]{\ensuremath{\mathcal{O}\left(#1\right)}}
% some simplifying commands
\newcommand{\eg}{e.g.}
%\newcommand{\eg}{{\it e.g.}} 
\newcommand{\ie}{i.e.}
\newcommand{\etal}{et al.}
\newcommand{\viz}{viz.}
\newcommand{\cf}{cf.}
\newcommand{\acite}[1]{{\bf(Add Citation: #1)}}
\newcommand{\E}{\mathcal{E}}
% derivative - d
\newcommand{\ud}[1]{\,\mathrm{d}{#1}\;}
% bold unit vector n-hat
\newcommand{\nhat}{\hat{\bf n}}
\newcommand{\tensor}[1]{\mathcal{#1}}
\renewcommand{\vec}[1]{\mathbf{#1}}
%Use for vectors of symbols (or use \pmb)
\newcommand{\vecsym}[1]{\boldsymbol{#1}}
\newcommand{\om}{\boldsymbol{\Omega}}

\newlength \figwidth
\setlength \figwidth {0.8\textwidth}
\captionsetup[subfigure]{width=0.4\textwidth} %trouble-maker
\captionsetup[subfigure]{font={footnotesize,stretch=1.1},skip=0pt}

 
 %LaTeX hyphenation
 \hyphenation{epi-thermal}

%Tikz flowchart parameters

\tikzstyle{block} = [rectangle, draw, fill=white, 
    text width=6em, text centered, rounded corners, minimum height=3em]
\tikzstyle{line} = [draw, thick, -latex']
 \tikzstyle{data} = [text width = 5em, text centered, node distance=1.6cm, inner sep=5pt] 
%

%Load the myriad packages
%\usepackage{amssymb,amsmath}
\usepackage{amsfonts}
\usepackage{amsmath}
\usepackage{textcomp}
%\usepackage{graphicx}
\usepackage{grffile}
\usepackage{tikz}
%\usepackage{subeqn} %for eqn 1a, 1b, etc
\usepackage{subfig}    % for multi-figure figures
%\usepackage[numbers, super]{natbib}
%\usepackage{pdflscape}
%\usepackage{rotating}
%\usepackage{grffile} %spaces in file names
%\usepackage{parskip}
%\usepackage[T1]{fontenc} %for sc and bf
%\usepackage{wasysym}
%\usepackage{bigstrut}
%\usepackage{hyperref}
%\usepackage{Float}
\usepackage{multirow}
\usepackage[numbers,sort&compress]{natbib}

%Tikz libraries
\usetikzlibrary{calc,trees,positioning,arrows,chains,shapes.geometric,%
    decorations.pathreplacing,decorations.pathmorphing,shapes,%
    matrix,shapes.symbols}

% Optional for code samples
\usepackage{listings}
\lstloadlanguages{Matlab}
\lstset{language=Matlab,commentstyle=\color{blue},keywordstyle=\color{red}}

%MCNP syntax highlighting
\definecolor{lightgrey}{rgb}{0.9,0.9,0.9}
\definecolor{darkgreen}{rgb}{0,0.6,0}
 
\lstdefinelanguage{MCNP}
{morekeywords={kcode,ksrc,mode,print,prdmp},
sensitive=false,
morecomment=[f]{c},
morecomment=[l]{$},
morestring=[b]",
numbers=left,
numberstyle =\tiny,
}

\lstset{
language=Matlab,
numbers=left,
numberstyle =\tiny,}

%placeholder cite
\newcommand{\CITE}{({\bf cite})}
%singular
\newcommand{\fref}[1]{Fig.~\ref{fig:#1}}
\newcommand{\Fref}[1]{Figure~\ref{fig:#1}}
\newcommand{\eref}[1]{Eq.~(\ref{eq:#1})}
\newcommand{\Eref}[1]{Equation~(\ref{eq:#1})}
\newcommand{\tref}[1]{Table~\ref{tab:#1}}
%plural
\newcommand{\frefs}[2]{Figs.~\ref{fig:#1} and \ref{fig:#2}}
\newcommand{\Frefs}[2]{Figures~\ref{fig:#1} and \ref{fig:#2}}
\newcommand{\erefs}[2]{Eqs.~(\ref{eq:#1}) and (\ref{eq:#2})}
\newcommand{\Erefs}[2]{Equations~(\ref{eq:#1}) and (\ref{eq:#2})}
\newcommand{\trefs}[2]{Tables~\ref{tab:#1} and \ref{tab:#2}}
%range
\newcommand{\frefss}[2]{Figs.~\ref{fig:#1} - \ref{fig:#2}}
\newcommand{\Frefss}[2]{Figures~\ref{fig:#1} - \ref{fig:#2}}
\newcommand{\erefss}[2]{Eqs.~(\ref{eq:#1}) - (\ref{eq:#2})}
\newcommand{\Erefss}[2]{Equations~(\ref{eq:#1}) - (\ref{eq:#2})}
\newcommand{\trefss}[2]{Tables~\ref{tab:#1} - \ref{tab:#2}}
%misc.
\newcommand{\nn}[1]{\ensuremath{^{#1}}} %[1] is # of commands
\newcommand{\keff}{\ensuremath{{k_\mathrm{eff}}}}
\newcommand{\kinf}{\ensuremath{{k_\infty}}}
\newcommand{\alphaT}{\ensuremath{{\alpha_{_T}}}}
\newcommand{\SN}{\ensuremath{{\text{S}_n}}}
\newcommand{\PN}{\ensuremath{{\text{P}_n}}}
\newcommand{\order}[1]{\ensuremath{\mathcal{O}\left(#1\right)}}
% some simplifying commands
\newcommand{\eg}{e.g.}
%\newcommand{\eg}{{\it e.g.}} 
\newcommand{\ie}{i.e.}
\newcommand{\etal}{et al.}
\newcommand{\viz}{viz.}
\newcommand{\cf}{cf.}
\newcommand{\acite}[1]{{\bf(Add Citation: #1)}}
\newcommand{\E}{\mathcal{E}}
% derivative - d
\newcommand{\ud}[1]{\,\mathrm{d}{#1}\;}
% bold unit vector n-hat
\newcommand{\nhat}{\hat{\bf n}}
\newcommand{\tensor}[1]{\mathcal{#1}}
\renewcommand{\vec}[1]{\mathbf{#1}}
%Use for vectors of symbols (or use \pmb)
\newcommand{\vecsym}[1]{\boldsymbol{#1}}
\newcommand{\om}{\boldsymbol{\Omega}}

\newlength \figwidth
\setlength \figwidth {0.8\textwidth}
\captionsetup[subfigure]{width=0.4\textwidth} %trouble-maker
\captionsetup[subfigure]{font={footnotesize,stretch=1.1},skip=0pt}

 
 %LaTeX hyphenation
 \hyphenation{epi-thermal}

%Tikz flowchart parameters

\tikzstyle{block} = [rectangle, draw, fill=white, 
    text width=6em, text centered, rounded corners, minimum height=3em]
\tikzstyle{line} = [draw, thick, -latex']
 \tikzstyle{data} = [text width = 5em, text centered, node distance=1.6cm, inner sep=5pt] 
%

%Load the myriad packages
%\usepackage{amssymb,amsmath}
\usepackage{amsfonts}
\usepackage{amsmath}
\usepackage{textcomp}
%\usepackage{graphicx}
\usepackage{grffile}
\usepackage{tikz}
%\usepackage{subeqn} %for eqn 1a, 1b, etc
\usepackage{subfig}    % for multi-figure figures
%\usepackage[numbers, super]{natbib}
%\usepackage{pdflscape}
%\usepackage{rotating}
%\usepackage{grffile} %spaces in file names
%\usepackage{parskip}
%\usepackage[T1]{fontenc} %for sc and bf
%\usepackage{wasysym}
%\usepackage{bigstrut}
%\usepackage{hyperref}
%\usepackage{Float}
\usepackage{multirow}
\usepackage[numbers,sort&compress]{natbib}

%Tikz libraries
\usetikzlibrary{calc,trees,positioning,arrows,chains,shapes.geometric,%
    decorations.pathreplacing,decorations.pathmorphing,shapes,%
    matrix,shapes.symbols}

% Optional for code samples
\usepackage{listings}
\lstloadlanguages{Matlab}
\lstset{language=Matlab,commentstyle=\color{blue},keywordstyle=\color{red}}

%MCNP syntax highlighting
\definecolor{lightgrey}{rgb}{0.9,0.9,0.9}
\definecolor{darkgreen}{rgb}{0,0.6,0}
 
\lstdefinelanguage{MCNP}
{morekeywords={kcode,ksrc,mode,print,prdmp},
sensitive=false,
morecomment=[f]{c},
morecomment=[l]{$},
morestring=[b]",
numbers=left,
numberstyle =\tiny,
}

\lstset{
language=Matlab,
numbers=left,
numberstyle =\tiny,}

%placeholder cite
\newcommand{\CITE}{({\bf cite})}
%singular
\newcommand{\fref}[1]{Fig.~\ref{fig:#1}}
\newcommand{\Fref}[1]{Figure~\ref{fig:#1}}
\newcommand{\eref}[1]{Eq.~(\ref{eq:#1})}
\newcommand{\Eref}[1]{Equation~(\ref{eq:#1})}
\newcommand{\tref}[1]{Table~\ref{tab:#1}}
%plural
\newcommand{\frefs}[2]{Figs.~\ref{fig:#1} and \ref{fig:#2}}
\newcommand{\Frefs}[2]{Figures~\ref{fig:#1} and \ref{fig:#2}}
\newcommand{\erefs}[2]{Eqs.~(\ref{eq:#1}) and (\ref{eq:#2})}
\newcommand{\Erefs}[2]{Equations~(\ref{eq:#1}) and (\ref{eq:#2})}
\newcommand{\trefs}[2]{Tables~\ref{tab:#1} and \ref{tab:#2}}
%range
\newcommand{\frefss}[2]{Figs.~\ref{fig:#1} - \ref{fig:#2}}
\newcommand{\Frefss}[2]{Figures~\ref{fig:#1} - \ref{fig:#2}}
\newcommand{\erefss}[2]{Eqs.~(\ref{eq:#1}) - (\ref{eq:#2})}
\newcommand{\Erefss}[2]{Equations~(\ref{eq:#1}) - (\ref{eq:#2})}
\newcommand{\trefss}[2]{Tables~\ref{tab:#1} - \ref{tab:#2}}
%misc.
\newcommand{\nn}[1]{\ensuremath{^{#1}}} %[1] is # of commands
\newcommand{\keff}{\ensuremath{{k_\mathrm{eff}}}}
\newcommand{\kinf}{\ensuremath{{k_\infty}}}
\newcommand{\alphaT}{\ensuremath{{\alpha_{_T}}}}
\newcommand{\SN}{\ensuremath{{\text{S}_n}}}
\newcommand{\PN}{\ensuremath{{\text{P}_n}}}
\newcommand{\order}[1]{\ensuremath{\mathcal{O}\left(#1\right)}}
% some simplifying commands
\newcommand{\eg}{e.g.}
%\newcommand{\eg}{{\it e.g.}} 
\newcommand{\ie}{i.e.}
\newcommand{\etal}{et al.}
\newcommand{\viz}{viz.}
\newcommand{\cf}{cf.}
\newcommand{\acite}[1]{{\bf(Add Citation: #1)}}
\newcommand{\E}{\mathcal{E}}
% derivative - d
\newcommand{\ud}[1]{\,\mathrm{d}{#1}\;}
% bold unit vector n-hat
\newcommand{\nhat}{\hat{\bf n}}
\newcommand{\tensor}[1]{\mathcal{#1}}
\renewcommand{\vec}[1]{\mathbf{#1}}
%Use for vectors of symbols (or use \pmb)
\newcommand{\vecsym}[1]{\boldsymbol{#1}}
\newcommand{\om}{\boldsymbol{\Omega}}

\newlength \figwidth
\setlength \figwidth {0.8\textwidth}
\captionsetup[subfigure]{width=0.4\textwidth} %trouble-maker
\captionsetup[subfigure]{font={footnotesize,stretch=1.1},skip=0pt}

 
 %LaTeX hyphenation
 \hyphenation{epi-thermal}

%Tikz flowchart parameters

\tikzstyle{block} = [rectangle, draw, fill=white, 
    text width=6em, text centered, rounded corners, minimum height=3em]
\tikzstyle{line} = [draw, thick, -latex']
 \tikzstyle{data} = [text width = 5em, text centered, node distance=1.6cm, inner sep=5pt] 